\chapter{Zaključak i budući rad}
		
	%	\textbf{\textit{dio 2. revizije}}\\
		
	%	 \textit{U ovom poglavlju potrebno je napisati osvrt na vrijeme izrade projektnog zadatka, koji su tehnički izazovi prepoznati, jesu li riješeni ili kako bi mogli biti riješeni, koja su znanja stečena pri izradi projekta, koja bi znanja bila posebno potrebna za brže i kvalitetnije ostvarenje projekta i koje bi bile perspektive za nastavak rada u projektnoj grupi.}
		
	%	 \textit{Potrebno je točno popisati funkcionalnosti koje nisu implementirane u ostvarenoj aplikaciji.}
		
		Projekt izrade web aplikacije "Zdravljem do znanja" u sklopu nastavnog kolegija „Programsko inženjerstvo“ uspješno je završen. Kada kažemo uspješno završen onda ne mislimo samo na uspješan konačni produkt projekta odnosno web aplikaciju, već na cijeli životni ciklus projekta.  Na projektu su radili sedam međusobno nepoznatih studenata koji nakon što su se upoznali i prilagodili rad jedni drugima, započeli su intenzivan rad na projektu. Podijelili smo posao na 3 skupine. Prvu skupinu su činili trojica studenta zadužena za frontend dio web aplikacije, drugu skupinu dvojica studenata zadužena za backend dio web aplikacije te posljednju, no ne i manje važnu skupinu dvojica studenata zadužena za izradu i vođenje dokumentacije.
		Projekt se odvijao kroz dvije faze. \\ \\
		 U prvoj fazi većina posla je bila usmjerena na dokumentaciju. Razlog tome je nužnost dokumentacije kao prethodni korak za ulazak u proces implementacije zahtjeva i izrade web aplikacije. Početna verzija dokumentacije je obuhvaćala popis funkcionalnih i nefunkcionalnih zahtjeva, oblikovanje i arhitekturu web aplikacije te nezaobilazni UML dijagrami obrazaca, sekvencijski dijagrami i dijagram razreda. Uz navedeno, u prvoj fazi je programski ostvarena i generička funkcionalnost registracije i prijave korisnika u sustav. \\ \\
		Druga faza projekta je obuhvaćala intenzivan rad na izradi web aplikacije i ostvarenju svih dokumentiranih zahtjeva, što su usklađeno i zajedničkim snagama realizirali skupina na frontend-u i backend-u . Osnovne funkcionalnosti koje su izrađene u sklopu web aplikacije su: registracija i prijava medicinskog djelatnika u sustav, registracija i prijava bolesnika u sustav, evidencija dolazaka bolesnika, izrada individualnog rasporeda tretmana za pojedinog bolesnika, unos kraja tretmana i zapažanja, izrada pdf dokumenta, pregled i unos tretmana i usluga te pregled aktivnosti medicinskih djelatnika i zauzeće uređaja. Također i u ovoj fazi aktivno, ali manje nego u prvoj fazi sudjeluje skupina za izradu dokumentacije. Konačna verzija dokumentacije proširuje  početnu dokumentacije ugrađujući dodatno UML dijagrame stanja, aktivnosti, komponenti i razmještaja te dokumentiranje cjelokupnog procesa ispitivanja programske potpore i puštanja web aplikacije u pogon. \\ \\
		Projekt nas je proveo kroz razne situacije, od početnog entuzijazma za rad pa blagog narušavanja komunikacije i slabljenja interesa do konačnog savladavanja svih prepreka i uzdizanja do želje da se projekt uspješno dovede do kraja. Projekt nam je donio veliko iskustvo za daljnji rad. Prikazao je stvarnu sliku grupnog rada na projektu, potaknuo nas je na učenje i upoznavanje sa novim tehnologijama - dotada nepoznatim radnim okvirima i programskim jezicima te ispitivanjem programske potpore. Upoznali smo koliko je važna dokumentacija u životnom ciklusu projekta. Bez dokumentacije projekt ne bilo moguće dovesti do uspješnog kraja. Dokumentacija nam je ponajprije olakšala i ubrzala rad na projektu. \\ \\ 
		Nakon ovakvog iskustva spremni smo se okušati u novim projektima i izazovima koji nas očekuju u budućem radu i karijeri nakon fakulteta.
		
		
		\eject 