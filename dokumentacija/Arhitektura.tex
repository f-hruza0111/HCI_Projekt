\chapter{Arhitektura i dizajn sustava}
		
		%\textbf{\textit{dio 1. revizije}}\\

		%\textit{ Potrebno je opisati stil arhitekture te identificirati: podsustave, preslikavanje na radnu platformu, spremišta podataka, mrežne protokole, globalni upravljački tok i sklopovsko-programske zahtjeve. Po točkama razraditi i popratiti odgovarajućim skicama:}
	\begin{comment}	
	\begin{itemize}
		\item 	\textit{izbor arhitekture temeljem principa oblikovanja pokazanih na predavanjima (objasniti zašto ste baš odabrali takvu arhitekturu)}
		\item 	\textit{organizaciju sustava s najviše razine apstrakcije (npr. klijent-poslužitelj, baza podataka, datotečni sustav, grafičko sučelje)}
		\item 	\textit{organizaciju aplikacije (npr. slojevi frontend i backend, MVC arhitektura) }		
	\end{itemize}
	\end{comment}
	Arhitekturu sustava definiramo kao struktura sustava koja sadrži elemente programa i njihova izvana vidljiva obilježja te odnose među njima. Arhitekturu sustava dijelimo u nekoliko podsustava i prikazujemo njihovu međusobnu komunikaciju. \\
		\begin{figure}[H]
		\includegraphics[scale=0.75]{slike/Arhitektura_sustava.PNG} %veličina slike u odnosu na originalnu datoteku i pozicija slike
		\centering
		\caption{Arhitektura sustava}
		\label{fig:promjene}
		\end{figure}
	
	
	Arhitektura našeg sustava odnosno web aplikacije sastoji se od dva osnovna dijela:
	\begin{itemize}
		\item Frontend
		\item Backend
	\end{itemize}
	\underbar{Frontend} predstavlja sve što krajnji korisnik vidi (prezentacija sadržaja, uređenje te raspored elemenata web aplikacije) i sve čime vrši interakciju (unos, odabir, obrazac, gumbi, i dr.). \\
	\underbar{Backend} predstavlja logiku web aplikacije odnosno kako web  aplikacija funkcionira te kako pristupa podacima iz baze podataka i datotečnog sustava.\\\\
	Frontend komunicira s backendom preko API-a. API (engl.\textit{Application Programming Interface}) predstavlja način na koji dva ili više računalnih programa međusobno komuniciraju. Kod web aplikacija se ta komunikacija temelji na HTTP zahtjevima i HTTP odgovorima.\\\\
	U backendu dijelu razlikujemo dva podsustava: web poslužitelj i baza podataka.\\
	\underbar{Web poslužitelj} je temeljna komponenta za rad web aplikacije. Omogućuje komunikaciju klijenta (web preglednika) s web aplikacijom. Komunikacija je ostvarena već spomenutim HTTP (engl. \textit{Hyper Text Transfer Protocol}) protokolom. \\
	\underbar{Web preglednik} je posrednik u komunikaciji između korisnika i web poslužitelja. Omogućuje korisniku prikaz web stranice sa svim njenim elementima.\\
	\underbar{Baza podataka} je skup međusobno povezanih podataka koji su tako organizirani da omoguće lakši pristup, pohranu i izmjenu podataka. Web poslužitelj komunicira s bazom podataka tako što šalje zahtjeve za podacima i dobiva podatke ukoliko postoje u bazi podataka.\\\\
	Frontend je ostvaren u programskom jeziku JavaScript u radnom okviru React, backend u programskom jeziku PHP u radnom okviru Symfony, a baza podataka pomoću Heroku PSQL baze. Razvojna okruženja koja koristimo su Visual Studio Code i IntelliJ IDEA.
	\\\\
	Oblikovanju arhitekture pristupili smo oblikovanjem od vrha prema dnu (engl. \textit{Top-down design}). Krenuli smo od osnovnih koncepata i funkcionalnosti web aplikacije do krajnjih detalja izvedbe. 
		
		\pagebreak		
		\section{Baza podataka}
			
			%\textbf{\textit{dio 1. revizije}}\\
			
		%\textit{Potrebno je opisati koju vrstu i implementaciju baze podataka ste odabrali, glavne komponente od kojih se sastoji i slično.}
		Baza podataka je skup međusobno povezanih podataka koji su pohranjeni i precizno organizirani kako bi omogućili lakše upravljanje podacima (pohrana, dohvat i izmjena podataka). Ona predstavlja sliku stvarnog organizacijskog sustava te strukturom olakšava modeliranje stvarnog svijeta. Relacija (tablica) je objekt baze podataka. Definiramo ju kao imenovanu dvodimenzionalnu tablicu čije imenovane stupce nazivamo atributi, a retke nazivamo n-torkama. U ovoj web aplikaciji izrađena je relacijska baza podataka pomoću Heroku PSQL baze i sastoji se od sljedećih entiteta:
		\begin{packed_item}
			\item User
			\item Staff
			\item Patient
			\item Appointment
			\item Treatment
		\end{packed_item}	
		
			\subsection{Opis tablica}
			

				%\textit{Svaku tablicu je potrebno opisati po zadanom predlošku. Lijevo se nalazi točno ime varijable u bazi podataka, u sredini se nalazi tip podataka, a desno se nalazi opis varijable. Svjetlozelenom bojom označite primarni ključ. Svjetlo plavom označite strani ključ}
				
				\textbf{User} Entitet sadrži osnovne informacije o korisniku web aplikacije. Atribute koje koristi su: id, name, surname, password, phone\_number, email i roles. Entitet je u vezi \textit{One-to-Zero-or-One}(1:0..1) s entitetom Staff pomoću atributa id te je u vezi \textit{One-to-Zero-or-One}(1:0..1) s entitetom Patient pomoću atributa id.
				
				\begin{longtblr}[
					label=none,
					entry=none
					]{
						width = \textwidth,
						colspec={|X[6,l]|X[6, l]|X[20, l]|}, 
						rowhead = 1,
					} %definicija širine tablice, širine stupaca, poravnanje i broja redaka naslova tablice
					\hline \multicolumn{3}{|c|}{\textbf{user}}	 \\ \hline[3pt]
					\SetCell{LightGreen}id & INT	&  	  Jedinstveni identifikator korisnika	\\ \hline
					name	& VARCHAR &   Ime korisnika	\\ \hline 
					surname & VARCHAR & Prezime korisnika  \\ \hline 
					password & VARCHAR &  Lozinka korisnika \\ \hline 
					phone\_number & VARCHAR & Broj telefona korisnika  \\ \hline
					email & VARCHAR &  Adresa elektroničke pošte \\ \hline 
					roles & JSON & Uloga/razina ovlasti korisnika  \\ \hline  
				\end{longtblr}
				
				\pagebreak
				\textbf{Staff} Entitet sadrži informacije o medicinskim djelatnicima zdravstvene ustanove. Atribute koje koristi su: id, user\_id, qualifications, specialization i availability. Entitet je u vezi \textit{Zero-or-One-to-One}(0..1:1) s entitetom User pomoću atributa user\_id, u vezi je  \textit{One-or-Many-to-One-or-Many}(1..N:1..N) s entitetom Patient pomoću atributa id te je u vezi \textit{One-to-One-or-Many}(1:1..N) s entitetom Appointment pomoću atributa id.
				
				\begin{longtblr}[
					label=none,
					entry=none
					]{
						width = \textwidth,
						colspec={|X[6,l]|X[6, l]|X[20, l]|}, 
						rowhead = 1,
					} %definicija širine tablice, širine stupaca, poravnanje i broja redaka naslova tablice
					\hline \multicolumn{3}{|c|}{\textbf{staff}}	 \\ \hline[3pt]
					\SetCell{LightGreen}id & INT	&  	  Jedinstveni identifikator djelatnika	\\ \hline
					\SetCell{LightBlue}user\_id	& INT &   Jedinstveni identifikator korisnika	\\ \hline 
					qualifications & VARCHAR & Stručna sprema djelatnika  \\ \hline 
					specialization & VARCHAR &  Specijalizacija djelatnika \\ \hline 
					availability & BOOLEAN & Dostupnost/nedostupnost djelatnika  \\ \hline
				\end{longtblr}
				
				\textbf{Patient} Entitet sadrži informacije o bolesnicima, tj. korisnicima usluga zdravstvene ustanove. Atribute koje koristi su: id, user\_id, doctor\_id, diagnosis, insurance\_id, is\_insured i services. Entitet je u vezi \textit{Zero-or-One-to-One}(0..1:1) s entitetom User pomoću atributa user\_id, u vezi je \textit{One-or-Many-to-One-or-Many}(1..N:1..N) s entitetom Staff pomoću atributa doctor\_id te je u vezi \textit{One-to-One-or-Many}(1:1..N) s entitetom Appointment pomoću atributa id.
				
				\begin{longtblr}[
					label=none,
					entry=none
					]{
						width = \textwidth,
						colspec={|X[6,l]|X[6, l]|X[20, l]|}, 
						rowhead = 1,
					} %definicija širine tablice, širine stupaca, poravnanje i broja redaka naslova tablice
					\hline \multicolumn{3}{|c|}{\textbf{patient}}	 \\ \hline[3pt]
					\SetCell{LightGreen}id & INT	&  	  Jedinstveni identifikator pacijenta	\\ \hline
					\SetCell{LightBlue}user\_id	& INT &   Jedinstveni identifikator korisnika	\\ \hline 
					\SetCell{LightBlue}doctor\_id & INT & Jedinstveni identifikator medicinskog djelatnika \\ \hline 
					diagnosis & VARCHAR &  Dijagnoza (povijest bolesti) pacijenta \\ \hline 
					insurance\_id & VARCHAR & MBO (matični broj osiguranika) \\ \hline
					is\_insured & BOOLEAN & Posjeduje li pacijent dodatno osiguranje \\ \hline
					services & JSON & Tretmani na koje je pacijent upućen \\ \hline				
				\end{longtblr}
				
				\pagebreak
				\textbf{Appointment} Entitet sadrži informacije o terminu tretmana za pacijenta. Atribute koje koristi su: id, patient\_id, staff\_id, service, date, attended i machine. Entitet je u vezi \textit{One-or-Many-to-One}(1..N:1) s entitetom Patient pomoću atributa patient\_id te je u vezi \textit{One-or-Many-to-One}(1..N:1) s entitetom Staff pomoću atributa staff\_id.
				
				\begin{longtblr}[
					label=none,
					entry=none
					]{
						width = \textwidth,
						colspec={|X[6,l]|X[6, l]|X[20, l]|}, 
						rowhead = 1,
					} %definicija širine tablice, širine stupaca, poravnanje i broja redaka naslova tablice
					\hline \multicolumn{3}{|c|}{\textbf{appointment}}	 \\ \hline[3pt]
					\SetCell{LightGreen}id & INT	&  	  Jedinstveni identifikator termina tretmana	\\ \hline
					\SetCell{LightBlue}patient\_id	& INT &   Jedinstveni identifikator pacijenta	\\ \hline 
					\SetCell{LightBlue}staff\_id	& INT &   Jedinstveni identifikator djelatnika	\\ \hline 
					service & VARCHAR & Usluge/tretmani pacijenta  \\ \hline 
					date & TIMESTAMP &  Datum i vrijeme termina tretmana \\ \hline 
					attended & BOOLEAN & Prisutnost/dolaznost pacijenta  \\ \hline
					machine & VARCHAR & Uređaj potreban za tretman \\ \hline			
				\end{longtblr}
				
				\textbf{Treatment} Entitet sadrži informacije o nazivu i opisu usluge/tretmana zdravstvene ustanove. Atribute koje koristi su: id, name i description. 
				
				\begin{longtblr}[
					label=none,
					entry=none
					]{
						width = \textwidth,
						colspec={|X[6,l]|X[6, l]|X[20, l]|}, 
						rowhead = 1,
					} %definicija širine tablice, širine stupaca, poravnanje i broja redaka naslova tablice
					\hline \multicolumn{3}{|c|}{\textbf{treatment}}	 \\ \hline[3pt]
					\SetCell{LightGreen}id & INT	&  	  Jedinstveni identifikator tretmana/usluge	\\ \hline
					name & VARCHAR & Naziv tretmana/usluge  \\ \hline 
					description & VARCHAR &  Opis tretmana/usluge \\ \hline 
				\end{longtblr}
				
				
			
			\subsection{Dijagram baze podataka}
				%\textit{ U ovom potpoglavlju potrebno je umetnuti dijagram baze podataka. Primarni i strani ključevi moraju biti označeni, a tablice povezane. Bazu podataka je potrebno normalizirati. Podsjetite se kolegija "Baze podataka".}
			\begin{figure}[H]
				\includegraphics[scale=0.55]{slike/E-R_dijagram_baze.PNG} %veličina slike u odnosu na originalnu datoteku i pozicija slike
				\centering
				\caption{E-R dijagram baze podataka}
				\label{fig:promjene}
			\end{figure}
			\eject
			
			
		\section{Dijagram razreda}
		
		%	\textit{Potrebno je priložiti dijagram razreda s pripadajućim opisom. Zbog preglednosti je moguće dijagram razlomiti na više njih, ali moraju biti grupirani prema sličnim razinama apstrakcije i srodnim funkcionalnostima.}\\
			
		%	\textbf{\textit{dio 1. revizije}}\\
			
			\begin{comment}
				content...	\textit{Prilikom prve predaje projekta, potrebno je priložiti potpuno razrađen dijagram razreda vezan uz \textbf{generičku funkcionalnost} sustava. Ostale funkcionalnosti trebaju biti idejno razrađene u dijagramu sa sljedećim komponentama: nazivi razreda, nazivi metoda i vrste pristupa metodama (npr. javni, zaštićeni), nazivi atributa razreda, veze i odnosi između razreda.}\\
			\end{comment}
			Svi razredi koji imaju sufiks "Controller" nasljeđuju razred AbstractController koji je dio Symfony radnog okvira te modelira kontroler komponente MVC arhitekture. Svaki od ovih razreda se bavi manipulacijom podataka te slanjem podataka na frontend kako bi se pogledi mogli ažurirati.
			Appointment razred modelira termin terapije te ima reference na djelatnika koje je voditelj te pacijenta kojemu je termin namijenjen. Staff predstavlja djelatnika ustanove. On ima referencu na razred User. Razred User predstavlja generaliziranog korisnika te služi za pridjeljivanje uloga korisnicima te čuva podatke potrebne za login.On ima referencu na 0 ili više objekata ApiToken.
			ApiToken služi autentifikaciju pri loginu. Razred Patient predstavlja pacijenta te ima referencu na User objekt te također na djelatnika koji ga je registrirao. Treatment enkapsulira informacije o pojedinoj usluzi koju ustanova nudi. PDFGenerator je razred koji služi za generiranje PDF dokumenta.
		
		
		
				\begin{figure}[H]
				\includegraphics[scale=0.4]{slike/classDiagram.png} %veličina slike u odnosu na originalnu datoteku i pozicija slike
				\centering
				\caption{Dijagram razreda}
				\label{fig:classDiagram}
				\end{figure}
			
			
		%	\textbf{\textit{dio 2. revizije}}\\			
			
			%\textit{Prilikom druge predaje projekta dijagram razreda i opisi moraju odgovarati stvarnom stanju implementacije}
			
			
			
			\eject
		
		\section{Dijagram stanja}
			
			
			%\textbf{\textit{dio 2. revizije}}\\
			
			%\textit{Potrebno je priložiti dijagram stanja i opisati ga. Dovoljan je jedan dijagram stanja koji prikazuje \textbf{značajan dio funkcionalnosti} sustava. Na primjer, stanja korisničkog sučelja i tijek korištenja neke ključne funkcionalnosti jesu značajan dio sustava, a registracija i prijava nisu. }
			
			Dijagram stanja prikazuje način korištenja aplikacije od strane medicinskog djelatnika odnosno navigaciju korisničkim sučeljem specifičnim za medicinskog djelatnika. Prijelazi između stanja događaju se pritiskom na određeni gumb u sučelju. Nakon autentifikacije, sustav generira početnu stranicu koja je u ovom slučaju početno stanje. Iz ovog stanja pomoću opcija navigacijske trake djelatnik može prijeći u neko novo stanje, odnosno neki drugi pogled. Ulaskom u stanje "Pregled svih aktivnosti", sustav se prvo nalazi u podstanju "Aktivnosti". Ovisno o odabranoj opciji, filtrira i prikazuje podatke vezane uz odabranu opciju, odnosno podstanje u kojem se sustav trenutno nalazi.
			U stanju "Popis bolesnika" djelatnik odabire opciju "Napravi raspored" te sustav prelazi u novo stanje u kojem iscrtava formu za izradu rasporeda. Pritiskom gumba "Dodaj" sustav dojavljuje poruku o uspjehu, a u suprotnom dojavljuje grešku korisniku te ostaje u istom stanju u oba slučaja. U stanju "Pregled svih termina" korisnik pritiskom na gumb "Evidentiraj dolazak" odlazi na formu za potvrdu evidencije dolaska bolesnika taj dan te se informacija o dolasku sprema u bazu podataka. U ovom stanju djelatnik također ima opciju evidentirati kraj terapije te čijim odabirom sustav prelazi u stanje "Izrada izvješća". U tom stanju sustav iscrtava formu za unos podataka izvješća te provjerava podatke. Pritiskom na gumb potvrde sustav prelazi u stanje "Generiranje izvješća". U ovom stanju sustav stvara objekt PDF generator koji generira PDF dokument izvješća te ga prikazuje u novom prozoru. Sustav nakon toga izlazi iz trenutnog stanja te provjerava je li polje Email bilo popunjeno. Ako postoji email u polju, sustav prelazi u stanje "Slanje Email-a" u kojem objekt Mailer generira i šalje poruku koja sadrži izvješće te prelazi u stanje "Izrada izvješća" i dojavljuje poruku od uspjehu. Ako ne postoji e-mail adresa, prelazi se u stanje "Izrada izvješća".
			
			\begin{figure}[H]
				\includegraphics[scale=0.29]{slike/stateDiagram.png} %veličina slike u odnosu na originalnu datoteku i pozicija slike
				\centering
				\caption{Dijagram stanja}
				\label{fig:stateDiagram}
			\end{figure}
			
			
			\eject 
		
		\section{Dijagram aktivnosti}
			
			%\textbf{\textit{dio 2. revizije}}\\
			
			% \textit{Potrebno je priložiti dijagram aktivnosti s pripadajućim opisom. Dijagram aktivnosti treba prikazivati značajan dio sustava.}
			
			Djelatnik odabire opciju za evidenciju kraja tretmana bolesniku. Sustavi interno već pamti podatke tako da ih nije potrebno dohvaćati. Aplikaciju prikazuje formu za izradu izvješća te ju djelatnik popunjava. Aplikacija stvara objekt PDF generator te mu prosljeđuje podatke o pacijentu i unesene podatke. PDF generator stvara izvješće te otvara izvješće u PDF obliku u novom prozoru. Aplikacija tada provjerava postoji li email u email polju forme te ako postoji šalje dokument na tu adresu i obavještava korisnika o uspjehu te proces završava. Ako je polje email prazno, proces se završava
			
			\begin{figure}[H]
				\includegraphics[scale=0.4]{slike/actionDiagram.png} %veličina slike u odnosu na originalnu datoteku i pozicija slike
				\centering
				\caption{Dijagram stanja}
				\label{fig:actionDiagram}
			\end{figure}
			
			\eject
		\section{Dijagram komponenti}
		
			%\textbf{\textit{dio 2. revizije}}\\
		
			% \textit{Potrebno je priložiti dijagram komponenti s pripadajućim opisom. Dijagram komponenti treba prikazivati strukturu cijele aplikacije.}
			Dijagram komponenti predstavlja UML dijagram koji omogućuje vizualizaciju organizacije implementiranih komponenti te njihove međusobne ovisnosti. Također prikazuje odnos programske potpore prema okolini. U našem komponentnom dijagramu prikazanom na slici 4.6 razlikujemo 3 osnovne komponente, a to su frontend web aplikacija, web aplikacija (sustav) i baza podataka.\\ \\
			Prvo sučelje je \underline{sučelje za dohvat HTML, CSS i JS datoteka}. Preko njega se poslužuju datoteke koje pripadaju frontend dijelu aplikacije. Router je glavna komponenta unutar sustava koja upravlja datotekama te određuje koja će se datoteka poslužiti na sučelje. Frontend dio sustava obuhvaća nekoliko JavaScript (JS) komponenti podijeljenih po načelu aktora i osnovnih aktivnosti koje pruža web aplikacija. Sve JS komponente ovise o React-ovim bibliotekama. \\ \\
			Drugo sučelje je \underline{sučelje za dohvat JSON podataka}. Preko njega se pristupa REST API komponenti koja je zadužena za posluživanje datoteka sa backend dijela aplikacije. Backend dio sustava je ostvaren kroz 3 međusobno povezane komponente Controller, Repository te Entity. \\ \\
			Pomoću Doctrine ORM dohvaćaju se podaci iz baze podataka te poslužuju dalje na korištenje komponentama backend dijela aplikacije.  React view preko navedenih sučelja komunicira sa web aplikacijom (sustavom) te ovisno o korisnikovim akcijama i zahtjevima osvježava prikaz i dohvaća nove podatke ili datoteke.
			
			\begin{figure}[H]
				\includegraphics[scale=0.36]{slike/componentDiagram.PNG} %veličina slike u odnosu na originalnu datoteku i pozicija slike
				\centering
				\caption{Dijagram komponenti web aplikacije}
				\label{fig:promjene}
			\end{figure}