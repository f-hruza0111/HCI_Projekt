\chapter{Opis projektnog zadatka}
		
	%	\textbf{\textit{dio 1. revizije}}\\
		
	%	\textit{Na osnovi projektnog zadatka detaljno opisati korisničke zahtjeve. Što jasnije opisati cilj projektnog zadatka, razraditi problematiku zadatka, dodati nove aspekte problema i potencijalnih rješenja. Očekuje se minimalno 3, a poželjno 4-5 stranica opisa.	Teme koje treba dodatno razraditi u ovom poglavlju su:}
	%	\begin{packed_item}
	%		\item \textit{potencijalna korist ovog projekta}
	%		\item \textit{postojeća slična rješenja (istražiti i ukratko opisati razlike u odnosu na zadani zadatak). Dodajte slike koja predočavaju slična rješenja.}
	%		\item \textit{skup korisnika koji bi mogao biti zainteresiran za ostvareno rješenje.}
	%		\item \textit{mogućnost prilagodbe rješenja }
	%		\item \textit{opseg projektnog zadatka}
	%		\item \textit{moguće nadogradnje projektnog zadatka}
	%	\end{packed_item}
		
	%	\textit{Za pomoć pogledati reference navedene u poglavlju „Popis literature“, a po potrebi konzultirati sadržaj na internetu koji nudi dobre smjernice u tom pogledu.}
		Ideja projekta je izrada web aplikacije "Fizikalna terapija" za potrebe zdravstvene ustanove "Zdravljem do znanja". \\ \\
		Djelatnosti koje obavlja ova ustanova su fizikalna terapija i rehabilitacija bolesnika nakon lakših ili težih povreda. Ustanova broji ukupno 15 djelatnika, od kojih su 9 fizikalni terapeuti, 3 medicinske sestre, 2 administratora te voditelj ustanove. Broj djelatnika nije konačan već se može mijenjati. \\ \\
		Zahtjev zdravstvene ustanove je učinkovit informacijski sustav koji omogućuje izradu rasporeda bolesnika na terapije koje ustanova pruža, praćenje podataka o bolesnicima i njihovom napretku u liječenju. Također sustav bi trebao omogućiti praćenje rada svih djelatnika ustanove. Pristup navedenim mogućnostima imaju samo registrirani djelatnici ustanove. Registrirani korisnici usluga zdravstvene ustanove imaju mogućnost uvida u vlastiti termin te popis i opis usluga/tretmana koje dobivaju. Neregistrirani korisnici mogu vidjeti samo pois i opis svih usluga/tretmana koje ustanova pruža.\\ \\
		Razlikujemo četiri vrste korisnika informacijskog sustava:
		\begin{packed_item}
			\item vlasnik sustava (administrator)
			\item voditelj ustanove
			\item medicinski djelatnici (fizioterapeuti i medicinske sestre)
			\item bolesnici (korisnici usluga)
		\end{packed_item}	
		\\\\
		\underbar{Vlasnik sustava (administrator)} upisuje sve podatke o uslugama/tretmanima te registrira medicinske djelatnike, tj. unosi podatke o medicinskim djelatnicima. Vlasnik sustava može, ali i ne mora biti zaposlenik ustanove.\\\\
		\parindent\underbar{Voditelj ustanove} upisuje sve podatke o uslugama/tretmanima, registrira medicinske djelatnike te ima mogućnost pregleda aktivnosti u ustanovi: pregled aktivnosti djelatnika, broj pacijenata po djelatniku i broj pacijenata po usluzi/tretmanu te zauzeće pojedinih uređaja. Sve aktivnosti može pregledavati u proizvoljnom vremenskom intervalu.\\\\
		\underbar{Medicinski djelatnik} može biti fizioterapeut ili medicinska sestra. Nakon što je registriran od strane administratora ili voditelja ustanove, može se prijaviti u sustav i pristupiti funkcionalnostima sustava. Funkcionalnosti su: izrada rasporeda pacijenata, registracija pacijenata, evidencija dolaska pacijenta i obavljenosti terapije te evidencija stanja pacijenta na kraju terapije. Također može kao i voditelj ustanove pristupiti pregledu broja pacijenata po usluzi/tretmanu, zauzeću pojedinih uređaja te broju pacijenata, ali samo za one pacijente za koje je registriran kao voditelj za određeni dan.\\\\
		\underbar{Bolesnici} se pri prvom dolasku u ustanovu registriraju od strane medicinskog djelatnika. Nakon registracije, bolesnik ima mogućnost prijave u sustav i pregleda vlastitog rasporeda tretmana, njihovog opisa te potvrde o dolaznosti na terapiju.
		\\\\
		Sustav omogućava istovremeni rad svih korisnika sustava te podržava unos hrvatskih dijakritičkih znakova.
		\\\\
		Za registraciju medicinskog djelatnika potrebni su sljedeći podaci:
		\begin{packed_item}
			\item ime
			\item prezime
			\item stručna sprema
			\item specijalizacija (ukoliko postoji)
			\item adresa elektroničke pošte
			\item broj telefona
			\item nedostupnost medicinskog djelatnika (bolovanje, godišnji odmor i dr.)
		\end{packed_item}
		\\\\
		Za registraciju bolesnika potrebni su sljedeći podaci:
		\begin{packed_item}
			\item ime
			\item prezime
			\item broj telefona
			\item adresa elektroničke pošte (ukoliko postoji)
			\item dijagnoza (iz liste dijagnoza bolesti prema MKB)
			\item MBO (matični broj osiguranika)
			\item dodatno osiguranje (da/ne)
			\item ime i prezime liječnika koji je uputio bolesnika na terapiju (iz liste liječnika)
			\item lista odabranih tretmana/usluga (na koje je bolesnik upućen)
		\end{packed_item}
		
		Pri prvom dolasku bolesnika u ustanovu, medicnski djelatnik koji ga zaprima izvršava registraciju bolesnika. Ukoliko je bolesnik već prije bio na nekom tretmanu, odnosno ima izvršenu registraciju, onda nije potrebna ponovna registracija već se podaci dohvaćaju iz baze podataka. Nakon registracije, isti medicinski djelatnik koji je zaprimio bolesnika izrađuje personalizirani raspored tretmana za bolesnika. Bitno je naglasiti sljedećih nekoliko uvjeta vezanih uz mogućnost izrade rasporeda.
		\begin{packed_enum}
			\item Radno vrijeme ustanove je 8 sati dnevno u razdoblju od 9:00 do 17:00 sati, od ponedjeljka do petka. Unutar tog razdoblja se vrše medicinske usluge pacijentima. Subotom, nedjeljom, blagdanima i praznicima je ustanova zatvorena.
			\item Ustanova posjeduje ograničeni broj resursa (uređaja i prostora) te je trajanje pojedine terapije vremenski unaprijed određeno. Svi uređaji mogu raditi paralelno i u isto vrijeme.
			\item Svaki bolesnik ima terapiju u trajanju od 10 dana i to uvijek u isto vrijeme u danu. Ne postoji mogućnost nadoknade u slučaju ne dolaska bolesnika na predviđeni termin.  
		\end{packed_enum}
		Potrebno je uskladiti izradu rasporeda bolesnika sa navedenim uvjetima, tj. poslovanjem zdravstvene ustanove.
		\\\\
		Prilikom svakodnevnog dolaska bolesnika na terapiju, medicinski djelatnik evidentira bolesnikov dolazak. Sustav automatski registrira po danima medicinskog djelatnika koji je prihvatio pojedinog bolesnika. 
		\\\\
		Nakon izvršenog desetog dana terapije medicinski djelatnik obavlja razgovor s bolesnikom te u sustav unosi stanje bolesnika nakon terapije (napredak, zadovoljstvo, aktualni problem, izjava bolesnika). Također postoji mogućnost unosa zapažanja medicinskog djelatnika u sustav. Nakon razgovora izrađuje se dokument koji sadrži sve podatke o bolesniku, tretmanu/usluzi, broju dolazaka, imenima djelatnika koji su zaprimili bolesnika po danima te podaci uneseni prilikom završnog razgovora. Stvara se dokument u pdf formatu, koji se onda ispisuje te ga bolesnik potpisuje. Dokument se pohranjuje u arhivu ustanove te se po želji bolesnika može poslati elektronička verzija dokumenta na adresu elektroničke pošte bolesnika. To je ujedno i završna faza tretmana/usluge. 
		\\\\
		Opasnost od Covid zaraze je još uvijek prisutna u svakodnevnom životu, stoga kao podsjetnik na zarazu sustav preuzima podatke s javno dostupnog servisa o broju zaraženih u Republici Hrvatskoj na taj dan i trend broja zaraza u određenom vremenskom intervalu. Svi navedeni podaci će se prikazati medicinskim djelatnicima prilikom prijave u sustav. 
		\eject	
	